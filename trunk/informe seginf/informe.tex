%%	SECCION documentclass																									 %%	
%%---------------------------------------------------------------------------%%
\documentclass[a4paper]{report}

%%---------------------------------------------------------------------------%%
%%	SECCION usepackage																											 %%	
%%---------------------------------------------------------------------------%%
\usepackage{amsmath, amsthm}
\usepackage[spanish,activeacute]{babel}
\usepackage{caratula}
\usepackage{a4wide}
\usepackage{hyperref}
\usepackage{fancyhdr}
\usepackage{graphicx} % Para el logo magico!
\usepackage{amssymb}
\usepackage{amsmath}
\usepackage[latin1]{inputenc}
\usepackage[dvipsnames,usenames]{color}
\usepackage{amsfonts}
\usepackage{highlight}
\usepackage{marvosym}
\usepackage{subfigure}
\usepackage{pdflscape}
\usepackage{color}
\usepackage{colortbl}
\usepackage{float}

%%---------------------------------------------------------------------------%%
%%	SECCION opciones																												 %%	
%%---------------------------------------------------------------------------%%
\parskip    = 11 pt
\headheight	= 13.1pt
\pagestyle	{fancy}
\definecolor{orange}{rgb}{1,0.5,0}

\addtolength{\headwidth}{1.0in}

\addtolength{\oddsidemargin}{-0.5in}
\addtolength{\textwidth}{1.0in}
\addtolength{\topmargin}{-0.5in}
\addtolength{\textheight}{0.7in}
\definecolor{goldenrod}{rgb}{0.85,0.65,0.13}

\newcounter{usCounter}
\setcounter{usCounter}{1}

\newcommand{\us}[4]{
{\setlength{\arrayrulewidth}{1mm}
\framebox{\parbox{16cm}{
{\textsl{\Large \underline{User story \arabic{usCounter}:}}}
\begin{itemize}
\item \underline{\textbf{Como:}} #1
\item \underline{\textbf{Quiero:}} #2
\item \underline{\textbf{De forma que:}} #3
\end{itemize}
\rule{16cm}{0.2mm}
\underline{\textbf{Story points:}} #4
} } }
\addtocounter{usCounter}{1}}

\newcommand{\specus}[3]{
{\setlength{\arrayrulewidth}{1mm}
\framebox{\parbox{16cm}{
\begin{itemize}
\item \underline{\textbf{Como:}} #1
\item \underline{\textbf{Quiero:}} #2
\item \underline{\textbf{De forma que:}} #3
\end{itemize}
} } }
}


%%---------------------------------------------------------------------------%%
%%	SECCION document	 %%	
%%---------------------------------------------------------------------------%%
\begin{document}
\renewcommand{\chaptername}{Parte }

%%---- Caratula -------------------------------------------------------------%%
\materia{Ingenieria de software 2 (2do cuatrimestre de 2009)}
\titulo{Trabajo Pr�ctico}

\integrante{Elizalde Victoria}{452/06}{kivielizalde@gmail.com}
\integrante{Gonzalez, Sergio}{481/06}{seges.ar@gmail.com}
\integrante{Mart'inez, Federico}{17/06}{federicoemartinez@gmail.com}
\integrante{Tleye, Sebastian}{732/05}{stleye@gmail.com}
%\grupo{Grupo ??}
\resumen{
%TODO
}


% TOC, usa estilos locos
\maketitle
\pagestyle{empty}
{
\fancypagestyle{plain}
    {
    \fancyhead{}
    \fancyfoot{}
    \renewcommand{\headrulewidth}{0.0pt}
    } % clear header and footer of plain page because of ToC
\tableofcontents
}

\newpage
% arreglos los estilos para el resto del documento, y
% reseteo los numeros de pagina para que queden bien
\pagenumbering{arabic}
\fancypagestyle{plain} {
    \fancyhead[LO]{Elizalde, Gonzalez, Mart�nez, Tleye}
    \fancyhead[C]{}
    \fancyhead[RO]{P\'agina \thepage\ de \pageref{LastPage}}
    \fancyfoot{}
    \renewcommand{\headrulewidth}{0.4pt}
}
\pagestyle{plain}
\chapter{Introducci�n}
A lo largo de este informe presentaremos el trabajo de implementaci�n realizado para la materia Seguridad de la Informaci�n.

El informe se organiza de la siguiente manera:
\begin{enumerate}
\item Consideraciones generales: Descripci�n general de la aplicaci�n: lenguaje de programaci�n utilizado, modelado de la base de datos, etc.
\item Sniffer: Descripci�n de la implementaci�n del sniffer
\item Reportes: Descripci�n de la herramienta para la generaci�n de reportes
\item Instalaci�n y uso: Documentaci�n sobre como armar el ambiente para utilizar el software y como usarlo
\end{enumerate}

\chapter{Consideraciones generales}
\section{Plataforma}
El trabajo fue realizado y probado en Linux Mint 7: Gloria, el cual esta basado en Ubuntu Jaunty. Creemos que el software deber�a funcionar en otras distribuciones Linux sin problemas, no asi en windows, ya que scapy (una libreria de la cual hablaremos mas adelante) no se encuentra disponible para windows (aunque existe un port\footnote{http://trac.secdev.org/scapy/wiki/WindowsInstallationGuide} que supuestamente brinda casi todas las posibilidades de scapy en windows, no lo probamos).

\section{Lenguaje de programaci�n}
Para la realizaci�n del proyecto utilizamos el lenguaje python, el cual varios miembros del grupo utilizamos. Consideramos que el desarrollo en este lenguaje es sencillo, lo cual permit�a que el resto del grupo puede familiarizarse r�pidamente con el, ya que no hab�a un lenguaje manejado por todos.

Las ventaja principal que presenta es justamente que permite una velocidad alta para el desarrollo, sin embargo como contra la performance no es comparable a la que podr�amos haber obtenido usando otro lenguaje como por ejemplo C++.

\section{Base de datos}
Para la realizaci�n del trabajo era necesario la utilizaci�n de una base de datos para poder almacenar el tr�fico sniffeado y luego poder analizarlo. Por una cuesti�n de simplicidad se utiliz� el motor sqlite, sin embargo es muy simple de cambiar, a fin de utilizar otro motor mas robusto como puede ser mySQL.

\subsection{SQLAlchemy}
Para facilitarnos la interacci�n con la base de datos decidimos utilizar SQLAlchemy, un ORM para python que es bastante flexible, ademas de simple de utilizar.

Definimos entonces las siguientes clases para hacer el mapeo a la base de datos:
\begin{itemize}
\item MensajeHTTP: Es la clase base de los mensajes HTTP. Posee las caracter�sticas generales de los mensajes, ya sean requests, o responses; como pueden ser los headers, el body, direcciones ip o los puertos.
\item RequestHTTP: Subclase de MensajeHTTP, guarda la uri, el m�todo y el identificador del response asociado (si corresponde)
\item ResponseHTTP: Subclase de MensajeHTTP, guarda el c�digo de respuesta y el mensaje de status
\item RequestNoHTTP: Clase que representa aquellos mensajes sniffeados que no corresponden a tr�fico HTTP (por ejemplo SSH, o HTTPS) y que provienen desde algun host hacia el proxy.
\item ResponseNoHTTP: Similar a la anterior, pero para los mensajes que provienen del proxy a un host de la red.
\end{itemize}

\begin{figure}[H]
\centering
\includegraphics[scale=1]{./figuras/persistencia.png}
\caption{Diagrama con las clases que mapean a la base de datos}
\end{figure}

\chapter{Sniffer}
\section{Sniffeo de la red}
La primera parte del trabajo, consisti� en la implementaci�n de un sniffer, el cual deb�a escuchar el tr�fico que se dirige a un proxy y guardar en una base de datos los mensajes HTTP y otros mensajes como ser SSH, SSL, o TLS.

\begin{figure}[H]
\centering
\includegraphics[scale=0.4]{./figuras/sniffer.png}
\caption{Esquema de la ubicaci�n del sniffer}
\end{figure}

Para la construcci�n del sniffer utilizamos la librer�a Scapy. Esta es una librer�a para python que brinda amplias facilidades para el manejo de paquetes a nivel TCP, Ethernet, ICMP y dem�s. Posee ademas otras caracter�sticas interesantes, como la capacidad de generar traceroutes.

Scapy provee una forma de sniffear tr�fico en una interfaz de red. Utilizando esto construimos nuestro sniffer, el cual es capaz de recibir callbacks que se invocan cada vez que se recibe un nuevo paquete. Esta funcionalidad nos permite construir alrededor del sniffer no solo la funcionalidad de capturar y guardar el tr�fico http, sino que permite ademas construir otras funcionalidades que se consideren �tiles. En nuestro caso utilizamos esta posibilidad de los callbacks para realizar dumps a archivos pcap o para ir mostrando por pantalla los paquetes que se iban capturando.

Otra caracter�stica positiva de construir al sniffer de esta manera, es que permite separar el proceso de sniffear del proceso de construir los paquetes HTTP y dem�s, para guardarlos en la base de datos.

El siguiente es un esquema que ilustra la estructura del sniffer:

\begin{figure}[H]
\centering
\includegraphics[scale=0.7]{./figuras/esquemaSniff.png}
\end{figure}

En el esquema vemos que utilizando las capacidades de scapy para hacer sniff logramos capturar los paquetes, que luego nuestro sniff pasa a sus suscriptores mediante los callbacks que estos  registran.

\section{Mensajes HTTP}
Cuando se detecta un paquete que llega al puerto del proxy, el mismo pasa a ser examinado. Del paquete se extrae el contenido (o capa RAW), si no hay contenido RAW (por ejemplo se trata de un mensaje TCP SYN) no se procesa. Si el contenido es un mensajeHTTP el mismo se persiste. Si no, se chequea que el mensaje sea el comienzo de un mensaje HTTP y si es as� se crea un estado para la cuadrupla propia de esa conexi�n TCP, donde se van a ir agregando el contenido de los paquetes que lleguen de esa conexi�n, hasta tener el mensaje HTTP completo. Es posible que el mensaje HTTP llegue partido porque TCP no se preocupa por eso sino que es responsabilidad de la aplicaci�n saber cuando terminan sus datos, y como nosotros estamos escuchando la red, vamos a tener que ser nosotros quienes rearmen el mensaje.

Entre los posibles mensajes que se pueden escuchar, hay requests cuyo m�todo  es CONNECT. Estos en general se usan antes de comenzar una conexi�n SSH, TLS, o SSL. Entonces cuando se detecta un CONNECT, se empieza a escuchar ese tr�fico para detectar si se trata de tr�fico no HTTP. En caso afirmativo el tr�fico de la conexi�n se persiste. Notar que a esta altura solo diferenciamos tr�fico HTTP, de tr�fico no HTTP. La discriminaci�n entre otros protocolos de aplicaci�n es posterior.

\section{Otras caracter�sticas de sniffer}
Ademas de sniffear tr�fico HTTP y persitirlo, la aplicaci�n brinda otras funcionalidades, que si bien no fueron pedidas, consideramos que pueden ser �tiles, entre ellas:

\begin{itemize}
\item Dumpeo de la sesi�n de captura a un archivo pcap. Esto es �til si se quiere analizar en otra herramienta las capturas realizadas. El archivo generado contiene todos los paquetes capturados, no solo lo que es tr�fico HTTP.
\item Simulaci�n de una sesi�n de captura a partir de un archivo pcap. El sniffer puede simular una captura en vivo mediante un archivo pcap. Esto es �til si se realizan capturas con otro sniffer como el wireshark, y se desea poder analizar a la misma con la herramienta de reportes que presentaremos mas adelante y que corresponde a la segunda parte del trabajo. Para leer archivos pcaps, tambi�n utilizamos scapy, que nos permite dado un archivo pcap obtener la lista de paquetes que hay en el, sin embargo scapy no preserva las fechas de captura de los paquetes, por lo cual la sesi�n queda guardada en la base de datos, como si se hubiera capturado en vivo en el momento en el que se corri� nuestro sniffer.
\end{itemize}

Ademas, dado que se utiliza un dise�o basado en callbacks, es f�cil extender las funcionalidades del sniffer, simplemente definiendo una clase y suscribiendo un m�todo de la misma que realice las manipulaciones deseada al paquete.

%TODO: tal vez un ejemplo de esto?

\section{Algunas limitaciones}
Nuestro sniffer presenta algunas limitaciones, creemos que las mas importantes son:

\subsection*{Performance}
Si bien creemos que python es un gran lenguaje y muy flexible, somos conscientes que su performance dista de la que se puede lograr programando en C, ademas pudimos averiguar que scapy puede perder paquetes si la carga es muy grande \footnote{http://www.secdev.org/projects/scapy/}. Si bien no pudimos probar al sniffer con una carga muy grande, creemos que es factible que la performance sea un limitante para su uso.

\subsection*{Consideraciones de seguridad}
B�sicamente el sniffer no fue hecho para correr en ambientes hostiles. As� por ejemplo creemos que es factible que un usuario malicioso logre generar una denegaci�n de servicio, generando la ca�da del sniffer. Esto se debe a que como comentamos antes, cuando llega un fragmento de mensaje HTTP se guarda cierta informaci�n de estado para poder armar todo el mensaje. 


\begin{figure}[H]
\centering
\includegraphics[scale=0.3]{./figuras/fragTCP.png}
\caption{Un mensaje HTTP puede venir fragmentado en varios mensajes TCP}
\end{figure}

Entonces un usuario malicioso puede enviar una gran cantidad de mensajes inconclusos y el sniffer, al guardar la informaci�n de estado de estos, termina agotando sus recursos. Esta situaci�n es similar a lo que ocurre con un ataque de \textit{SYN flooding} en TCP. 

\begin{figure}[H]
\centering
\includegraphics[scale=0.4]{./figuras/atraganta.png}
\caption{Al enviar muchos paquetes inconclusos, un usuario malicioso podr�a lograr agotar los recursos del sniffer}
\end{figure}


Esta vulnerabilidad no fue probada, sin embargo la consideramos altamente factible de explotar.





%TODO: hablar de que se puede hacer un spoof attack
%TODO: hablar de que solo debe ser usado por el admin
\chapter{Generaci�n de reportes}
\section{ReporTool}
La segunda parte del trabajo consisti� en la realizaci�n de reportes sobre el tr�fico capturado por el sniffer.
Nosotros construimos una herramienta que denominamos reporTool. La misma permite realizar los distintos an�lisis de tr�fico (reportes) generando una salida en formato pdf o en formato html mediante una interfaz gr�fica.

\begin{figure}[H]
\centering
\includegraphics[scale=0.4]{./figuras/Pantallazo.png}
\caption{Captura de pantalla de reporTool}
\end{figure}

En las siguientes secciones daremos una breve descripci�n de los reportes realizados por nosotros. Posteriormente explicaremos el modo interactivo que presenta la herramienta y finalmente mostraremos como es posible agregar reportes propios para ser corridos con reporTool.

\section{Reportes}
Los siguientes son los reportes incluidos con reporTool

\subsection{Reporte de horario laboral}
Este reporte permite analizar el tr�fico discriminandolo seg�n un periodo laboral (d�as de la semana, y rango horario). De esta manera el reporte puede mostrar que usuarios usaron Internet fuera de dichos horarios y que porcentaje de su actividad estuvo en infracci�n. El reporte puede mostrar todos los pedidos que se realizaron fuera de hora o solo mostrar el porcentaje en infracci�n.

Ademas puede mostrar la distribuci�n del uso de Internet a lo largo de las distintas horas del d�a.

\subsection{Reporte por listas negras (blacklists)}
Este reporte permite especificar en un archivo una lista negra con dominios. El reporte analiza el tr�fico buscando visitas a alguno de esos dominios. Es en realidad una familia de reportes, ya que es posible utilizar varias instancias del mismo utilizando distintas listas negras, con distintos tipos de dominios, por ejemplo sitios de apuestas, de warez, etc. 

Si bien reporTool trae unas listas negras para ciertas categor�as, es f�cil modificar esas listas, o reemplazarlas por otras, crear un reporte nuevo que trabaje con listas de otras categor�as.

Estas son las listas negras incluidas con reporTool:
%TODO: listar las blackslists que se incluyen en la entrega

Los siguientes son sitios donde es posible descargar listas negras de distintas categor�as:
\begin{itemize}
\item \verb_http://squidguard.mesd.k12.or.us/_
\item \verb_http://www.shallalist.de/_
\item \verb<http://cri.univ-tlse1.fr/documentations/cache/squidguard_en.html#contrib<
\item \verb_http://urlblacklist.com/?sec=download_
\end{itemize}

Este reporte puede generar gr�ficos de cantidad de infracciones por usuarios, que porcentaje del tr�fico por usuario y total esta en infracci�n, cuales son los sitios de la lista negra mas visitados, entre otras cosas.

\subsection{Reporte de tr�fico AJAX}
%TODO: sergio completa esto

\subsection{Reporte por contenido (content-type)}
En este reporte podemos ver el tipo de contenido del tr�fico, b�sicamente hicimos seis separaciones distintas (audio, video, aplicaci�n, multipart, imagen y texto). En el reporte podemos ver, por ejemplo, la cantidad de tr�fico de tipo imagen y la cantidad de tr�fico que no es imagen. Tambi�n tenemos la cantidad de tr�fico de imagen separada por usuario para poder ver m�s en detalle qu� hace cada usuario.

Si bien se podr�an hacer varias separaciones mas (como por ejemplo discriminar por tr�fico de tipo jpeg o pdf) estas separaciones generales nos parecieron las m�s interesantes para mostrar en el reporte.
Igualmente es factible hacer algunas de estas consultas mediante el modo interactivo (o extendiendo el reporte).

\subsection{Reporte por tr�fico de nivel de aplicaci�n}
En este reporte se retoma el an�lisis del tr�fico seg�n protocolo de la capa de aplicaci�n (ya al guardar los requests en la base de datos, los distinguimos entre tr�fico http y no http).

Aprovechando esta divisi�n, analizamos por un lado cada tipo de tr�fico. Del tr�fico http se analiza �nicamente el volumen y los IPs involucrados (para poder hacer el reporte de tr�fico discriminado por usuario, y el de tr�fico total). Por otro lado, buscamos 3 protocolos distintos en el tr�fico no http: SSL, TLS (el sucesor de SSL) y SSH.

La forma de distinguir paquetes SSL y TLS es esencialmente la misma: se revisa el segundo y tercer byte de cada paquete, buscando una serie de n�meros (los distintos n�meros de versi�n del protocolo). Si el protocolo es SSL tendr� un cierto n�mero de versi�n, si es TLS otro (hay mas de un n�mero de versi�n de TLS).

\begin{figure}[H]
\centering
\includegraphics[scale=0.4]{./figuras/TLS.png}
\caption{Un paquete SSL / TLS}
\end{figure}

En cambio, para distinguir si se trata de un paquete SSH, hay que buscar el string "SSH-versi�nProtocolo-versi�nSoftware", que est� en texto en claro. Adem�s de este string podemos obtener el software utilizado. Uno de los reportes muestra los clientes SSH utilizados: nuestra suposici�n es que dada la topolog�a de la red (habiendo que pasar a trav�s de un proxy http) se puede hacer SSH hacia afuera de la red y no hacia adentro. Entonces, analizamos que software se utiliz� mirando los request no http (y no lo responses) y obtenemos cual es el software que est� permitiendo al infractor hacer SSH hacia afuera de la red.

\subsection{Reporte de tr�fico global}
%TODO: sergio completa esto, o si queres lo hago yo, pero comete un paty

\subsection{Reporte de seguimiento}
Este reporte permite elegir hasta 5 sitios y analizar la evoluci�n de la cantidad de visitas y de tr�fico mes a mes para estos sitios, a lo largo de un periodo de tiempo determinado.

\subsection{Reporte por heur�stica}
Este reporte puede pensarse de alguna forma como un complemento para el reporte por blacklists. El mismo utiliza un archivo donde se definen ciertas palabras clave (por ejemplo, palabras sobre apuestas: P�ker, Black Jack, casino, etc) que son buscadas en las requests y en los responses. El reporte muestra luego para que usuarios se encontraron matches, as� como cuales fueron los principales matches.

Las listas negras solo encuentran requests a los dominios definidos en ellos, entonces es factible que un usuario pueda visitar alguna p�gina que no est� en la lista pero que deber�a estarlo, o que incluso saltee el chequeo de las listas negras (por ejemplo navegando con el cache de Google). La b�squeda heur�stica permite detectar estas cosas, al costo de posiblemente dar falsos positivos sobre el contenido de las visitas. Sin embargo puede ser una herramienta �til para detectar usos indebidos.

\section{Modo interactivo}
Ademas de la posibilidad de realizar reportes, reporTool provee un mecanismo interactivo para realizar consultas a la base de datos. En vez de brindar una interfaz para simplemente ejecutar SQL en la base de datos, lo cual es posible con cualquier cliente de bases de datos; brindamos una consola python con ciertos elementos de alto nivel, brindados por el ORM que permiten interactuar con la base de datos.

Para obtener una sesi�n, se debe utilizar el comando get\_session. Una vez que tenemos una sesi�n, podemos realizar consultas sobre las clases que mapean a la base de datos (presentadas anteriormente) mediante la interfaz que provee sqlalchemy. 

Para mas informaci�n se puede consultar:

\begin{itemize}
\item \verb<http://www.sqlalchemy.org/docs/05/session.html#querying<
\item \verb<http://www.sqlalchemy.org/docs/05/reference/orm/query.htmlz<
\end{itemize}

A continuaci�n presentamos un ejemplo de como obtener las requests originadas en la IP 10.0.2.17:

\framebox{\begin{minipage}[t][1\totalheight]{1\columnwidth}%
\noindent
\ttfamily
\shorthandoff{"}\\
\hlstd{Python\ }\hlnum{2.6.2\ }\hlstd{}\hlsym{(}\hlstd{release26}\hlsym{{-}}\hlstd{maint}\hlsym{,\ }\hlstd{Apr\ }\hlnum{19\ 2009}\hlstd{}\hlsym{,\ }\hlstd{}\hlnum{01}\hlstd{}\hlsym{:}\hlstd{}\hlnum{56}\hlstd{}\hlsym{:}\hlstd{}\hlnum{41}\hlstd{}\hlsym{)}\hspace*{\fill}\\
\hlstd{}\hlsym{{[}}\hlstd{GCC\ }\hlnum{4.3.3}\hlstd{}\hlsym{{]}\ }\hlstd{on\ linux2\hspace*{\fill}\\
Type\ }\hlstr{``help''}\hlstd{}\hlsym{,\ }\hlstd{}\hlstr{``copyright''}\hlstd{}\hlsym{,\ }\hlstd{}\hlstr{``credits''}\hlstd{\ }\hlkwa{or\ }\hlstd{}\hlstr{``license''}\hlstd{\ }\hlkwa{for\ }\hlstd{more\ information}\hlsym{.}\hspace*{\fill}\\
\hlstd{}\hlsym{$>$$>$$>$\ }\hlstd{s\ }\hlsym{=\ }\hlstd{}\hlkwd{get\textunderscore session}\hlstd{}\hlsym{()}\hspace*{\fill}\\
\hlstd{}\hlsym{$>$$>$$>$\ }\hlstd{q\ }\hlsym{=\ }\hlstd{s}\hlsym{.}\hlstd{}\hlkwd{query}\hlstd{}\hlsym{(}\hlstd{RequestHTTP}\hlsym{)}\hspace*{\fill}\\
\hlstd{}\hlsym{$>$$>$$>$\ }\hlstd{reqs\ }\hlsym{=\ }\hlstd{q}\hlsym{.}\hlstd{}\hlkwb{filter}\hlstd{}\hlsym{(}\hlstd{RequestHTTP}\hlsym{.}\hlstd{ipOrigen\ }\hlsym{==\ }\hlstd{}\hlstr{``10.0.2.17''}\hlstd{}\hlsym{).}\hlstd{}\hlkwd{all}\hlstd{}\hlsym{()}\hlstd{}\hspace*{\fill}\\
\mbox{}
\normalfont
\shorthandon{"}

\end{minipage}}

Tambi�n es posible ejecutar consultas sobre la base con sintaxis SQL, mediante la utilizaci�n del motor, sin embargo para utilizar SQL es conveniente usar el cliente propio del motor de la base de datos. El siguiente ejemplo permite obtener todas las requests (el resultado es un ResultProxy):

\framebox{\begin{minipage}[t][1\totalheight]{1\columnwidth}%

\noindent
\ttfamily
\shorthandoff{"}\\
\hlstd{Python\ }\hlnum{2.6.2\ }\hlstd{}\hlsym{(}\hlstd{release26}\hlsym{{-}}\hlstd{maint}\hlsym{,\ }\hlstd{Apr\ }\hlnum{19\ 2009}\hlstd{}\hlsym{,\ }\hlstd{}\hlnum{01}\hlstd{}\hlsym{:}\hlstd{}\hlnum{56}\hlstd{}\hlsym{:}\hlstd{}\hlnum{41}\hlstd{}\hlsym{)}\hspace*{\fill}\\
\hlstd{}\hlsym{{[}}\hlstd{GCC\ }\hlnum{4.3.3}\hlstd{}\hlsym{{]}\ }\hlstd{on\ linux2\hspace*{\fill}\\
Type\ }\hlstr{``help''}\hlstd{}\hlsym{,\ }\hlstd{}\hlstr{``copyright''}\hlstd{}\hlsym{,\ }\hlstd{}\hlstr{``credits''}\hlstd{\ }\hlkwa{or\ }\hlstd{}\hlstr{``license''}\hlstd{\ }\hlkwa{for\ }\hlstd{more\ information}\hlsym{.}\hspace*{\fill}\\
\hlstd{}\hlsym{$>$$>$$>$\ }\hlstd{con\ }\hlsym{=\ }\hlstd{engine}\hlsym{.}\hlstd{}\hlkwd{connect}\hlstd{}\hlsym{()}\hspace*{\fill}\\
\hlstd{}\hlsym{$>$$>$$>$\ }\hlstd{con}\hlsym{.}\hlstd{}\hlkwd{execute}\hlstd{}\hlsym{(}\hlstd{}\hlstr{``select\ {*}\ from\ requests''}\hlstd{}\hlsym{)}\hlstd{}\hspace*{\fill}\\
\mbox{}
\normalfont
\shorthandon{"}


\end{minipage}}

Para mas informaci�n puede consultarse:

\verb<http://www.sqlalchemy.org/docs/05/reference/sqlalchemy/connections.html<

\section{Agregado de nuevos reportes}
ReporTool fue pensado para ser f�cilmente extensible, es decir que sea f�cil poder agregar nuevos reportes definidos por el usuario.

Para poder definir un nuevo reporte, es necesario definir una clase que herede de Reporte. Esta misma debe implementar el m�todo ejecutar que recibe un desde y un hasta, que son las fechas entre las cuales debe analizarse el tr�fico.

Dicho m�todo debe devolver un string latex, que es el que se utilizar� para generar el pdf o el html correspondiente al reporte.

Una vez definida esta clase, debe importarse en reporTool, crear una instancia, envolverla con un Configurador y agregar al configurador a la lista de configuradores.
 
Para ilustrar un poco mejor esto, presentaremos un ejemplo muy sencillo de como se define y se agrega un nuevo reporte.

\subsection{Definici�n de la clase}
Creamos un nuevo archivo llamado reporteTrucho.py, en el ponemos:

\framebox{\begin{minipage}[t][1\totalheight]{1\columnwidth}%

\noindent
\ttfamily
\shorthandoff{"}\\
\hlstd{}\hlkwa{from\ }\hlstd{reporte\ }\hlkwa{import\ }\hlstd{Reporte}\hspace*{\fill}\\
\hlkwa{from\ }\hlstd{latex\ }\hlkwa{import\ }\hlstd{LatexFactory}\hspace*{\fill}\\
\hlkwa{from\ }\hlstd{enthought}\hlsym{.}\hlstd{traits}\hlsym{.}\hlstd{api\ }\hlkwa{import\ }\hlstd{}\hlsym{{*}}\hspace*{\fill}\\
\hlstd{}\hlkwa{from\ }\hlstd{enthought}\hlsym{.}\hlstd{traits}\hlsym{.}\hlstd{ui}\hlsym{.}\hlstd{api\ }\hlkwa{import\ }\hlstd{}\hlsym{{*}}\hspace*{\fill}\\
\hlstd{}\hspace*{\fill}\\
\hlkwa{class\ }\hlstd{}\hlkwd{ReporteTrucho}\hlstd{}\hlsym{(}\hlstd{Reporte}\hlsym{):}\hspace*{\fill}\\
\hlstd{\hspace*{\fill}\\
}\hlstd{\ \ \ \ }\hlstd{flagTrucho\ }\hlsym{=\ }\hlstd{}\hlkwd{Bool}\hlstd{}\hlsym{(}\hlstd{}\hlkwa{True}\hlstd{}\hlsym{)}\hspace*{\fill}\\
\hlstd{}\hlstd{\ \ \ \ }\hlstd{rangoTrucho\ }\hlsym{=\ }\hlstd{}\hlkwd{Range}\hlstd{}\hlsym{(}\hlstd{}\hlnum{1}\hlstd{}\hlsym{,}\hlstd{}\hlnum{10}\hlstd{}\hlsym{,}\hlstd{}\hlnum{5}\hlstd{}\hlsym{)}\hspace*{\fill}\\
\hlstd{\hspace*{\fill}\\
}\hlstd{\ \ \ \ }\hlstd{}\hlkwa{def\ }\hlstd{}\hlkwd{ejecutar}\hlstd{}\hlsym{(}\hlstd{self}\hlsym{,}\hlstd{desde}\hlsym{,}\hlstd{hasta}\hlsym{):}\hspace*{\fill}\\
\hlstd{}\hlstd{\ \ \ \ \ \ \ \ }\hlstd{l\ }\hlsym{=\ }\hlstd{}\hlkwd{LatexFactory}\hlstd{}\hlsym{()}\hspace*{\fill}\\
\hlstd{}\hlstd{\ \ \ \ \ \ \ \ }\hlstd{l}\hlsym{.}\hlstd{}\hlkwd{chapter}\hlstd{}\hlsym{(}\hlstd{}\hlstr{``Reporte\ Trucho''}\hlstd{}\hlsym{)}\hspace*{\fill}\\
\hlstd{}\hlstd{\ \ \ \ \ \ \ \ }\hlstd{l}\hlsym{.}\hlstd{}\hlkwd{texto}\hlstd{}\hlsym{(}\hlstd{}\hlstr{``\%s''}\hlstd{}\hlsym{\%}\hlstd{self}\hlsym{.}\hlstd{flagTrucho}\hlsym{)}\hspace*{\fill}\\
\hlstd{}\hlstd{\ \ \ \ \ \ \ \ }\hlstd{}\hlkwa{return\ }\hlstd{l}\hlsym{.}\hlstd{}\hlkwd{generarOutput}\hlstd{}\hlsym{()}\hlstd{}\hspace*{\fill}\\
\mbox{}
\normalfont
\shorthandon{"}
        
\end{minipage}}

Analicemos un poco el c�digo:
La primera l�nea importa la clase Reporte, para que el reporte que queremos definir pueda heredar de dicha clase.

Luego se importa la clase LatexFactory que brinda algunas funciones para poder generar c�digo latex mas f�cilmente.

Las l�neas 3 y 4 son necesarias para importar la librer�a traits que permite generar luego de forma autom�tica la GUI para modificar el reporte.

Luego comienza la definici�n de la clase. Las dos primeras lineas de la definici�n, establecen dos atributos que va a tener el reporte, un flag booleano que por defecto vale True y un rango que va del 1 al 10 y que por defecto comienza en 5. Estos atributos se podr�n luego modificar mediante la interfaz gr�fica.

Luego se define el m�todo ejecutar como dijimos anteriormente. En este caso se devuelve un cap�tulo latex con el t�tulo \textit{Reporte trucho} y un texto que muestra el valor del flag.

Ya tenemos definida la clase del nuevo reporte, vamos a agregarlo. En rerpoTool.py agregamos las siguientes lineas:

\framebox{\begin{minipage}[t][1\totalheight]{1\columnwidth}%

\noindent
\ttfamily
\shorthandoff{"}\\
\hlstd{}\hlslc{\#\#\#\#\#\#\#\ Imports\ de\ los\ distintos\ reportes\ \#\#\#\#\#\#\#}\hspace*{\fill}\\
\hlstd{}\hlkwa{from\ }\hlstd{horarioLaboral\ }\hlkwa{import\ }\hlstd{FueraDeHorario}\hspace*{\fill}\\
\hlkwa{from\ }\hlstd{blackList\ }\hlkwa{import\ }\hlstd{ListaNegra}\hspace*{\fill}\\
\hlkwa{from\ }\hlstd{ajax\ }\hlkwa{import\ }\hlstd{Ajax}\hspace*{\fill}\\
\hlkwa{from\ }\hlstd{contentType\ }\hlkwa{import\ }\hlstd{ContentType}\hspace*{\fill}\\
\hlkwa{from\ }\hlstd{nonHTTP\ }\hlkwa{import\ }\hlstd{NonHTTP}\hspace*{\fill}\\
\hlkwa{from\ }\hlstd{evolucion\ }\hlkwa{import\ }\hlstd{EvolucionMensual}\hspace*{\fill}\\
\hlkwa{from\ }\hlstd{heuristica\ }\hlkwa{import\ }\hlstd{Heuristica}\hspace*{\fill}\\
\hlkwa{from\ }\hlstd{traficoEnGral\ }\hlkwa{import\ }\hlstd{TraficoEnGral}\hspace*{\fill}\\
\hspace*{\fill}\\
\hlkwa{from\ }\hlstd{reporteTrucho\ }\hlkwa{import\ }\hlstd{ReporteTrucho\ }\hlslc{\#Agregamos\ esta\ linea}\hlstd{}\hspace*{\fill}\\
\mbox{}
\normalfont
\shorthandon{"}

\end{minipage}}

De esta manera, importamos nuestro reporte, para que este disponible.

\framebox{\begin{minipage}[t][1\totalheight]{1\columnwidth}%

\noindent
\ttfamily
\shorthandoff{"}\\
\hlstd{}\hlslc{\#\#\#\#\#\#\#\#\#\#\#\#\#\#\#\#\#\#\#\#\#\#\#\#\#\#\#\#\#\#\#\#\#\#\#\#\#\#\#\#\#\#\#\#\#\#\#\#\#\#\#\#\#\#\#}\hspace*{\fill}\\
\hlstd{}\hlslc{\#\#}\hlstd{\ \ \ \ \ \ }\hlslc{Otros\ reportes\ definidos\ por\ el\ usuario}\hlstd{\ \ \ \ \ \ \ }\hlslc{\#}\hspace*{\fill}\\
\hlstd{}\hlslc{\#\#\#\#\#\#\#\#\#\#\#\#\#\#\#\#\#\#\#\#\#\#\#\#\#\#\#\#\#\#\#\#\#\#\#\#\#\#\#\#\#\#\#\#\#\#\#\#\#\#\#\#\#\#\#}\hspace*{\fill}\\
\hlstd{rt\ }\hlsym{=\ }\hlstd{}\hlkwd{ReporteTrucho}\hlstd{}\hlsym{()}\hspace*{\fill}\\
\hlstd{configDePrueba\ }\hlsym{=\ }\hlstd{}\hlkwd{Configurador}\hlstd{}\hlsym{(}\hlstd{script\ }\hlsym{=\ }\hlstd{rt}\hlsym{,\ }\hlstd{nombre}\hlsym{=}\hlstd{}\hlstr{``Reporte\ Trucho''}\hlstd{}\hlsym{,}\hlstd{descripcion\ }\hlsym{=\ }\hlstd{}\hlstr{``Soy\ un\ reporte\ muy\ trucho''}\hlstd{}\hlsym{)}\hlstd{}\hspace*{\fill}\\
\mbox{}
\normalfont
\shorthandon{"}

\end{minipage}}

En este c�digo, creamos la instancia de reporte, y lo wrappeamos en un configurador. Un configurador es lo que permite mostrar el nombre del reporte y los botones para seleccionar al reporte y para configurarlo.

\framebox{\begin{minipage}[t][1\totalheight]{1\columnwidth}%

\noindent
\ttfamily
\shorthandoff{"}\\
\hlstd{}\hlslc{\#\#\#\#\#\#\#\#\#\#\#\#\#\#\#\#\#\#\#\#\#\#\#\#\#\#\#\#\#\#\#\#\#\#\#\#\#\#\#\#\#\#\#\#\#\#\#\#\#\#\#\#\#\#}\hspace*{\fill}\\
\hlstd{configuradores}\hlsym{={[}}\hlstd{c}\hlsym{,}\hlstd{c1}\hlsym{,}\hlstd{c2}\hlsym{,}\hlstd{c3}\hlsym{,}\hlstd{c4}\hlsym{,}\hlstd{c5}\hlsym{,}\hlstd{c6}\hlsym{,}\hlstd{c7}\hlsym{,}\hlstd{c8}\hlsym{,}\hlstd{c9}\hlsym{,}\hlstd{c10}\hlsym{,}\hlstd{c11}\hlsym{,}\hlstd{c12}\hlsym{{]}}\hspace*{\fill}\\
\hlstd{}\hspace*{\fill}\\
\hspace*{\fill}\\
\hlslc{\#\#\ Registrar\ en\ esta\ lista\ los\ configuradores\ definidos\ por\ el\ usuario}\hspace*{\fill}\\
\hlstd{configuradores}\hlsym{.}\hlstd{}\hlkwd{append}\hlstd{}\hlsym{(}\hlstd{configDePrueba}\hlsym{)}\hlstd{}\hspace*{\fill}\\
\mbox{}
\normalfont
\shorthandon{"}

\end{minipage}}

Aqu� agregamos el configurador a la lista de configuradores de reporTool.

Finalmente si ejecutamos reporTool, vemos que el reporte ya esta listo para ejecutarse:

\begin{figure}[H]
\centering
\includegraphics[scale=0.5]{./figuras/reporteAgregado.png}
\end{figure}

De esta forma el reporte queda agregado. Si se quiere agregar un reporte para una sesi�n nada mas, se puede utilizar la consola del modo interactivo. Sin modificar el c�digo de reporTool usarse la funci�n agregarReporte que recibe una instancia de reporte (en el ejemplo una instancia de ReporteTrucho) y el nombre que se le quiere dar al reporte, y lo agrega a la lista de reportes de la interfaz gr�fica. Este es un modo r�pido de ir probando distintos reportes, los cuales se puede, incluso, definir en la consola.

Veamos como usamos esto para utilizar al reporteTrucho que definimos antes:

\framebox{\begin{minipage}[t][1\totalheight]{1\columnwidth}%
\noindent
\ttfamily
\shorthandoff{"}\\
\hlstd{Python\ }\hlnum{2.6.2\ }\hlstd{}\hlsym{(}\hlstd{release26}\hlsym{{-}}\hlstd{maint}\hlsym{,\ }\hlstd{Apr\ }\hlnum{19\ 2009}\hlstd{}\hlsym{,\ }\hlstd{}\hlnum{01}\hlstd{}\hlsym{:}\hlstd{}\hlnum{56}\hlstd{}\hlsym{:}\hlstd{}\hlnum{41}\hlstd{}\hlsym{)}\hspace*{\fill}\\
\hlstd{}\hlsym{{[}}\hlstd{GCC\ }\hlnum{4.3.3}\hlstd{}\hlsym{{]}\ }\hlstd{on\ linux2\hspace*{\fill}\\
Type\ }\hlstr{``help''}\hlstd{}\hlsym{,\ }\hlstd{}\hlstr{``copyright''}\hlstd{}\hlsym{,\ }\hlstd{}\hlstr{``credits''}\hlstd{\ }\hlkwa{or\ }\hlstd{}\hlstr{``license''}\hlstd{\ }\hlkwa{for\ }\hlstd{more\ information}\hlsym{.}\hspace*{\fill}\\
\hlstd{}\hlsym{$>$$>$$>$\ }\hlstd{}\hlkwa{from\ }\hlstd{reporteTrucho\ }\hlkwa{import\ }\hlstd{ReporteTrucho}\hspace*{\fill}\\
\hlsym{$>$$>$$>$\ }\hlstd{r\ }\hlsym{=\ }\hlstd{}\hlkwd{ReporteTrucho}\hlstd{}\hlsym{()}\hspace*{\fill}\\
\hlstd{}\hlsym{$>$$>$$>$\ }\hlstd{}\hlkwd{agregarReporte}\hlstd{}\hlsym{(}\hlstd{r}\hlsym{,\ }\hlstd{}\hlstr{``Reporte\ trucho''}\hlstd{}\hlsym{)}\hlstd{}\hspace*{\fill}\\
\mbox{}
\normalfont
\shorthandon{"}

\end{minipage}}

Con solo hacer esto, ya tenemos un nuevo reporte disponible.

\section{Algunas consideraciones de seguridad sobre ReporTool}
ReporTool es una herramienta pensada para ser usada por un usuario administrador, ya que la misma tiene acceso a la base de datos, y si la pudiera usar cualquier usuario, podr�a modificarla o borrarla a su antojo.

Otra consideraci�n que hay que hacer, es que al identificar a los usuarios por su IP un usuario malicioso podr�a inculpar a otro usuario inocente. Para esto no tendr�a mas que generar requests a sitios prohibidos por la blacklist spoofeando la IP de la v�ctima. Entonces al correr reporTool, en el reporte aparecer� que la v�ctima estuvo accediendo a sitios indebidos, cuando en verdad eso es falso.




\chapter{Instalaci�n y uso}
A continuaci�n daremos una gu�a para realizar la instalaci�n del sniffer y de reporTool. En ambos casos, presentaremos los pasos para un sistema operativo similar a Ubuntu, con python 2.6 ya instalado (por defecto en ubuntu).

\section{Instalaci�n del sniffer}
\begin{enumerate}
\item Instalar easy\_install para python:

\framebox{\begin{minipage}[t][1\totalheight]{1\columnwidth}%
\noindent
\ttfamily
\shorthandoff{"}\\
\hlstd{}\hlsym{$>$\ }\hlstd{sudo\ apt}\hlsym{{-}}\hlstd{get\ }\hlkwc{install\ }\hlstd{python}\hlsym{{-}}\hlstd{setuptools\ python}\hlsym{{-}}\hlstd{dev\ build}\hlsym{{-}}\hlstd{essential}\hspace*{\fill}\\
\mbox{}
\normalfont
\shorthandon{"}
\end{minipage}}

\item Instalar scapy:
 
\framebox{\begin{minipage}[t][1\totalheight]{1\columnwidth}%
\noindent
\ttfamily
\shorthandoff{"}\\
\hlstd{}\hlsym{$>$\ }\hlstd{sudo\ easy\textunderscore install\ scapy}\hspace*{\fill}\\
\mbox{}
\normalfont
\shorthandon{"}
\end{minipage}}


Si esto falla, instala scapy con apt-get (minimo la versi�n 2):

\framebox{\begin{minipage}[t][1\totalheight]{1\columnwidth}%
\noindent
\ttfamily
\shorthandoff{"}\\
\hlstd{}\hlsym{$>$\ }\hlstd{sudo\ apt}\hlsym{{-}}\hlstd{get\ }\hlkwc{install\ }\hlstd{scapy}\hspace*{\fill}\\
\mbox{}
\normalfont
\shorthandon{"}
\end{minipage}}


\item Instalar sqlAlchemy (ORM para la base de datos):

\framebox{\begin{minipage}[t][1\totalheight]{1\columnwidth}%
\noindent
\ttfamily
\shorthandoff{"}
\hlstd{}\hlsym{$>$\ }\hlstd{sudo\ easy\textunderscore install\ sqlalchemy}\hspace*{\fill}\\
\mbox{}
\normalfont
\shorthandon{"}
\end{minipage}}

\item Instalar dpkt (Armado y manipulaci�n de paquetes para varios protocolos), para ello, extraerlo, pararse en el directorio y ejecutar:

\framebox{\begin{minipage}[t][1\totalheight]{1\columnwidth}%
\noindent
\ttfamily
\shorthandoff{"}
\hlstd{}\hlsym{$>$\ }\hlstd{sudo\ python\ setup.py\ }\hlkwc{install}\hlstd{}\hspace*{\fill}\\
\mbox{}
\normalfont
\shorthandon{"}
\end{minipage}}

\item En la versi�n de scapy que bajamos nosotros hay un bug que impide cargar correctamente un archivo pcap. Nosotros hicimos un bugfix de eso, y se encuentra en la carpeta bugfix del CD. Para utilizarlo hay que reemplazar el archivo \verb_/var/lib/python-support/python2.6/sccapy/utils.py_ por el que brindamos nosotros.

\end{enumerate}

\section{Uso del sniffer}
Antes de empezar a sniffear hay que crear la base de datos para que el sniffer pueda guardar los paquetes. Para eso basta con ejecutar:

\framebox{\begin{minipage}[t][1\totalheight]{1\columnwidth}%
\noindent
\ttfamily
\shorthandoff{"}
\hlstd{}\hlsym{$>$\ }\hlstd{python\ persistencia.py}\hspace*{\fill}\\
\mbox{}
\normalfont
\shorthandon{"}
\end{minipage}}

Una vez hecho esto, ya se puede correr el sniffer con:

\framebox{\begin{minipage}[t][1\totalheight]{1\columnwidth}%
\noindent
\ttfamily
\shorthandoff{"}
\hlstd{}\hlsym{$>$\ }\hlstd{sudo\ python\ sniff.py}\hspace*{\fill}\\
\mbox{}
\normalfont
\shorthandon{"}
\end{minipage}}

Las opciones que soporta el sniffer son:
\begin{itemize}
\item -p nroPuerto (--port nroPuerto): Setea el puerto del proxy al valor nroPuerto. Por defecto es 8080
\item -i IP (--ip IP): Setea la ip del proxy al valor IP. Por defecto se ignora la ip, y se asume que todo trafico al puerto del proxy, va para el proxy.
\item -d file (--dump file): Guardar la sesi�n de sniffeo en un archivo pcap de nombre file.
\item -v (--verbose): Modo verbose, se muestra un resumen de cada paquete (a nivel tcp) que captura el sniffer.
\item -f file (--from-file file): Realizar la sesi�n simulando que se sniffean los paquetes que estan en el archivo pcap file
\item -h (--help): Mostrar ayuda, con las posibles opciones que tiene el sniffer
\end{itemize}

Una vez lanzado el sniffer continua corriendo, hasta que se lo interrumpa con \textit{ctrl+c}

\chapter{Conclusiones}

Este trabajo nos permiti� poner la pr�ctica algunos conceptos vistos en clase y de esta forma comprender mejor su funcionamiento, ya que es necesario siempre complementar la te�rica con la practica. Vimos un mejor acercamiento a los sniffer, ya que de alguna forma programamos uno de ellos, vimos un poco mas a bajo nivel el funcionamiento de SSH por ejemplo, ya que tuvimos que analizar su encabezado para identificar que tipo de version era. Tambi�n utilizamos herramientas de sniffeo, que nos permiti� manejarnos de forma mas fluida con archivos pcap, entre otras cosas; lo cual creemos que fue algo muy positivo.

Otro punto interesante, pero que puede estar fuera de los temas de la materia, fue la utilizaci�n del lenguaje $Python$ para realizar el trabajo practico. Esto nos permiti� avanzar en el conocimiento de dicho lenguaje y descubrir nuevas cosas que se pueden hacer con el.

A modo de conclusi�n queremos comentar que nos gust� mucho hacer este trabajo practico. Creemos que fue un buen cierre para una materia que disfrutamos, y que ademas nos permiti� aprender cosas que desconoc�amos totalmente y que creemos son importantes para nuestra formaci�n como profesionales.

\label{LastPage}
\end{document}