\chapter{Instalaci�n y uso}
A continuaci�n daremos una gu�a para realizar la instalaci�n del sniffer y de reporTool. En ambos casos, presentaremos los pasos para un sistema operativo similar a Ubuntu, con python 2.6 ya instalado (por defecto en ubuntu).

\section{Instalaci�n del sniffer}
\begin{enumerate}
\item Instalar easy\_install para python:

\framebox{\begin{minipage}[t][1\totalheight]{1\columnwidth}%
\noindent
\ttfamily
\shorthandoff{"}\\
\hlstd{}\hlsym{$>$\ }\hlstd{sudo\ apt}\hlsym{{-}}\hlstd{get\ }\hlkwc{install\ }\hlstd{python}\hlsym{{-}}\hlstd{setuptools\ python}\hlsym{{-}}\hlstd{dev\ build}\hlsym{{-}}\hlstd{essential}\hspace*{\fill}\\
\mbox{}
\normalfont
\shorthandon{"}
\end{minipage}}

\item Instalar scapy:
 
\framebox{\begin{minipage}[t][1\totalheight]{1\columnwidth}%
\noindent
\ttfamily
\shorthandoff{"}\\
\hlstd{}\hlsym{$>$\ }\hlstd{sudo\ easy\textunderscore install\ scapy}\hspace*{\fill}\\
\mbox{}
\normalfont
\shorthandon{"}
\end{minipage}}


Si esto falla, instala scapy con apt-get (minimo la versi�n 2):

\framebox{\begin{minipage}[t][1\totalheight]{1\columnwidth}%
\noindent
\ttfamily
\shorthandoff{"}\\
\hlstd{}\hlsym{$>$\ }\hlstd{sudo\ apt}\hlsym{{-}}\hlstd{get\ }\hlkwc{install\ }\hlstd{scapy}\hspace*{\fill}\\
\mbox{}
\normalfont
\shorthandon{"}
\end{minipage}}


\item Instalar sqlAlchemy (ORM para la base de datos):

\framebox{\begin{minipage}[t][1\totalheight]{1\columnwidth}%
\noindent
\ttfamily
\shorthandoff{"}
\hlstd{}\hlsym{$>$\ }\hlstd{sudo\ easy\textunderscore install\ sqlalchemy}\hspace*{\fill}\\
\mbox{}
\normalfont
\shorthandon{"}
\end{minipage}}

\item Instalar dpkt (Armado y manipulaci�n de paquetes para varios protocolos), para ello, extraerlo, pararse en el directorio y ejecutar:

\framebox{\begin{minipage}[t][1\totalheight]{1\columnwidth}%
\noindent
\ttfamily
\shorthandoff{"}
\hlstd{}\hlsym{$>$\ }\hlstd{sudo\ python\ setup.py\ }\hlkwc{install}\hlstd{}\hspace*{\fill}\\
\mbox{}
\normalfont
\shorthandon{"}
\end{minipage}}

\item En la versi�n de scapy que bajamos nosotros hay un bug que impide cargar correctamente un archivo pcap. Nosotros hicimos un bugfix de eso, y se encuentra en la carpeta bugfix del CD. Para utilizarlo hay que reemplazar el archivo \verb_/var/lib/python-support/python2.6/sccapy/utils.py_ por el que brindamos nosotros.

\end{enumerate}

\section{Uso del sniffer}
Antes de empezar a sniffear hay que crear la base de datos para que el sniffer pueda guardar los paquetes. Para eso basta con ejecutar:

\framebox{\begin{minipage}[t][1\totalheight]{1\columnwidth}%
\noindent
\ttfamily
\shorthandoff{"}
\hlstd{}\hlsym{$>$\ }\hlstd{python\ persistencia.py}\hspace*{\fill}\\
\mbox{}
\normalfont
\shorthandon{"}
\end{minipage}}

Una vez hecho esto, ya se puede correr el sniffer con:

\framebox{\begin{minipage}[t][1\totalheight]{1\columnwidth}%
\noindent
\ttfamily
\shorthandoff{"}
\hlstd{}\hlsym{$>$\ }\hlstd{sudo\ python\ sniff.py}\hspace*{\fill}\\
\mbox{}
\normalfont
\shorthandon{"}
\end{minipage}}

Las opciones que soporta el sniffer son:
\begin{itemize}
\item -p nroPuerto (--port nroPuerto): Setea el puerto del proxy al valor nroPuerto. Por defecto es 8080
\item -i IP (--ip IP): Setea la ip del proxy al valor IP. Por defecto se ignora la ip, y se asume que todo trafico al puerto del proxy, va para el proxy.
\item -d file (--dump file): Guardar la sesi�n de sniffeo en un archivo pcap de nombre file.
\item -v (--verbose): Modo verbose, se muestra un resumen de cada paquete (a nivel tcp) que captura el sniffer.
\item -f file (--from-file file): Realizar la sesi�n simulando que se sniffean los paquetes que estan en el archivo pcap file
\item -h (--help): Mostrar ayuda, con las posibles opciones que tiene el sniffer
\end{itemize}

Una vez lanzado el sniffer continua corriendo, hasta que se lo interrumpa con \textit{ctrl+c}